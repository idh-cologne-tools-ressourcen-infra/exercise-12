\documentclass{scrartcl}
%example document created with TeXworks
\date{04.02.2020}
\author{pjansse5}
\title{scientific LATEX example}
\begin{document}
\maketitle
\tableofcontents

\section{Intro}
THis is an example Document created with LATEX.
\section{Main}
From Wikipedia, the free encyclopedia

LaTeX is a software system for document preparation.[2] When writing, the writer uses plain text as opposed to the formatted text found in "What You See Is What You Get" word processors like Microsoft Word, LibreOffice Writer and Apple Pages. The writer uses markup tagging conventions to define the general structure of a document (such as article, book, and letter), to stylise text throughout a document (such as bold and italics), and to add citations and cross-references. A TeX distribution such as TeX Live or MiKTeX is used to produce an output file (such as PDF or DVI) suitable for printing or digital distribution.

LaTeX can be used as a standalone document preparation system, or as an intermediate format. In the latter role, for example, it is sometimes used as part of a pipeline for translating DocBook and other XML-based formats to PDF. The typesetting system offers programmable desktop publishing features and extensive facilities for automating most aspects of typesetting and desktop publishing, including numbering and cross-referencing of tables and figures, chapter and section headings, the inclusion of graphics, page layout, indexing and bibliographies.[6]

Like TeX, LaTeX started as a writing tool for mathematicians and computer scientists, but even from early in its development, it has also been taken up by scholars who needed to write documents that include complex math expressions or non-Latin scripts, such as Arabic,[7] Devanagari and Chinese.[8]

LaTeX is intended to provide a high-level, descriptive markup language that accesses the power of TeX in an easier way for writers. In essence, TeX handles the layout side, while LaTeX handles the content side for document processing. LaTeX comprises a collection of TeX macros and a program to process LaTeX documents, and because the plain TeX formatting commands are elementary, it provides authors with ready-made commands for formatting and layout requirements such as chapter headings, footnotes, cross-references and bibliographies.
\subsection{Formula example}
$c^{2}=\frac{ x^{2} }{5}+\sqrt{ a^{2}+b^{2} }$
\subsection{Table Example}
\begin{tabular}[t]{rl}
Data Type & Example\\
\hline
Integer & 1\\
Character & A\\
Boolean & True\\
\end{tabular}
\end{document}
