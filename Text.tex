\title{ISAAC NEWTON}
\section{Isaac Newton}
Sir Isaac Newton [ˌaɪzək ˈnjuːtən] (* 25. Dezember 1642jul./ 4. Januar 1643greg. in Woolsthorpe-by-Colsterworth in Lincolnshire; † 20. März 1726jul./ 31. März 1727greg. in Kensington)[1] war ein englischer Naturforscher und Verwaltungsbeamter. In der Sprache seiner Zeit, die zwischen natürlicher Theologie, Naturwissenschaften, Alchemie und Philosophie noch nicht scharf trennte, wurde Newton als Philosoph bezeichnet.

Isaac Newton ist der Verfasser der Philosophiae Naturalis Principia Mathematica, in denen er mit seinem Gravitationsgesetz die universelle Gravitation beschrieb und die Bewegungsgesetze formulierte, womit er den Grundstein für die klassische Mechanik legte. Fast gleichzeitig mit Gottfried Wilhelm Leibniz entwickelte Newton die Infinitesimalrechnung. Er verallgemeinerte das binomische Theorem mittels unendlicher Reihen auf beliebige reelle Exponenten. Bekannt ist er auch für seine Leistungen auf dem Gebiet der Optik: die von ihm verfochtene Teilchentheorie des Lichtes und die Erklärung des Lichtspektrums.

Aufgrund seiner Leistungen, vor allem auf den Gebieten der Physik und Mathematik (siehe Geschichte der Physik, Geschichte der Mathematik), gilt Sir Isaac Newton als einer der bedeutendsten Wissenschaftler aller Zeiten. Die Principia Mathematica werden als eines der wichtigsten wissenschaftlichen Werke eingestuft.

Eine Sammlung von Schriften, im Bestand der National Library of Israel, zu theologischen und alchemistischen Themen wurde 2015 von der UNESCO zum Weltdokumentenerbe erklärt.[2]

%%%%%%%%%%%%%%%%%%%%%%%%%%%%%%%


\section{Jugend}
Newtons gleichnamiger Vater Isaac Newton, ein erfolgreicher Schafzüchter und Inhaber des Titels Lord of the Manor, starb drei Monate vor der Geburt seines Sohnes. 1646 heiratete seine Mutter Hannah Ayscough zum zweiten Mal. Sie zog zu ihrem Ehemann Barnabas Smith, der Pfarrer in der nahen Gemeinde North Witham war, und Isaac blieb bei seiner Großmutter Margery Ayscough in Woolsthorpe.[3][4] Newton empfand über diese Vernachlässigung zeitlebens Bitterkeit, und er kam auch nicht mit seinem Großvater James Ayscough klar. Dieser hinterließ ihm nichts, als er 1653 starb und Newton erwähnte ihn später nie mehr. Als er mit 19 Jahren seine Sünden auflistete, war darunter auch der Wunsch, das Haus seiner Mutter und seines Stiefvaters Smith anzuzünden.[4] Nach dem Tod seines Stiefvaters 1653 kehrte seine Mutter nach Woolsthorpe zurück und Newton lebte kurz mit ihr, seiner Großmutter und den drei Kindern aus der Ehe seiner Mutter mit Smith. Als er bald darauf die Kings School in Grantham besuchte, eine öffentliche Schule (Free Grammar School) 5 Meilen von Woolthorpe, wohnte er in Grantham bei einer Familie Clark. Am Schulunterricht war er nach den Schulberichten wenig interessiert, soll aber zu Hause Vergnügen an mechanischen Basteleien gefunden haben. Seine nunmehr wohlhabende Mutter, eine Gutsbesitzerin, holte ihn versuchsweise von der Schule, damit er die Verwaltung ihres Vermögens übernahm, es zeigte sich aber, dass er dafür kein Talent und Interesse hatte.[4] Newtons Onkel William Ayscough überzeugte die Mutter, dass Newton studieren sollte, und Newton besuchte ab 1660 wieder die Schule in Grantham, wobei er diesmal beim Schulleiter Stokes wohnte und mehr Lerneifer zeigte. Möglicherweise kam er damals schon mit Euklids Elementen in Berührung, es gibt aber keinen sicheren Beleg dafür, dass dies vor seinem Studium 1663 geschah.[4]


%%%%%%%%%%%%%%%%%%%%%%%%%%%%%%%


\section{Studium}
Am 5. Juni 1661 begann er am Trinity College in Cambridge zu studieren, einem College, das schon sein Onkel besuchte. Er war – trotz des Vermögens seiner Mutter – ein Sizar, das heißt sein finanzieller Unterhalt wurde teilweise vom College übernommen. Dafür musste er als Diener für andere Studenten arbeiten. Möglicherweise war ein entfernter Verwandter und Fellow des Trinity College, Humphrey Babington, sein Patron.[4]

Er studierte zunächst mit der Absicht, Jurist zu werden. Ab dem dritten Studienjahr hatte er aber mehr Freiheiten in den Studienfächern. Damals war in Cambridge die Lehre von Aristoteles und die spätscholastische Schule der Cambridger Platoniker tonangebend, das bedeutet qualitative Naturphilosophie anstelle quantitativer Untersuchungen im Sinne von Galilei. Newtons Notizen aus der Studienzeit, die er Quaestiones quaedam philosophicae (Einige philosophische Fragen) betitelte, zeigen den Einfluss von Descartes’ mechanistisch-dualistischem Denken, Gassendis atomistischen Vorstellungen und Henry Mores platonisch-hermetischen Ansichten. Weiter studierte er Thomas Hobbes und Robert Boyle.[4] Obwohl sie radikal unterschiedlich sind, beeinflussten die Anschauungen der Mechanisten bzw. Hermetiker fortan Newtons Denken und bildeten – in ihrer Spannung – das Grundthema seiner Laufbahn als Naturphilosoph. Den Quaestiones stellte er allerdings den Spruch voran, dass Aristoteles und Plato seine Freunde wären, sein bester Freund wäre aber die Wahrheit.[4] Er studierte auch Galileo Galilei und Johannes Kepler (Optik).

Ab Ende 1663 begann er sich auch für Mathematik zu interessieren, las Euklids Elemente in der Ausgabe von Isaac Barrow (1630–1677), William Oughtred (Clavis mathematica), die Geometrie von Descartes und das Buch von Frans van Schooten darüber, die Ausgabe der Gesammelten Werke von François Viète von Frans van Schooten (mit Anhängen seiner Schüler Johan de Witt, Johan Hudde, Hendrick van Heuraet) und die Algebra von John Wallis, die auch schon erste Ansätze zur Analysis enthielt und Newton unmittelbar zu eigenen Arbeiten anregte. Barrow war 1663 Fellow des Trinity College geworden, Newton kam aber wohl erst ein paar Jahre später in näherem Kontakt zu ihm auf mathematischem Gebiet.[4]

Am 28. April 1664 wurde er Scholar, und im April 1665 erhielt er den Bachelor-Abschluss. Sein eigentlicher Durchbruch als Mathematiker und Naturwissenschaftler erfolgte, als die Universität im Sommer 1665 wegen der Großen Pest geschlossen wurde und er an seinen Wohnort Woolthorpe zurückkehrte, wo er die nächsten zwei Jahre bis zur Wiedereröffnung der Universität in relativer wissenschaftlicher Isolation verbrachte.

Nach seinem eigenen Bezeugen in den Quaestiones hatte er in den Jahren 1665/1666 seine ersten weitreichenden Ideen, die ihn auf die Spur seiner drei großen Theorien führten: der Infinitesimalrechnung (in Newtons Terminologie Theorie der Fluxionen), der Theorie des Lichts und der Gravitationstheorie. Wie weit er mit seinen theoretischen Ansätzen in dieser frühen Zeit schon war, ist unklar.[5] Die Veröffentlichung seiner Lehren auf diesen Gebieten bzw. Zirkulation seiner diesbezüglichen Manuskripte erfolgte erst viel später.
\end{itemize}
\end{itemize}
\section{Forschung in Naturwissenschaft und Philosophie}


%%%%%%%%%%%%% Optik %%%%%%%%%%%%%%%%%%

\subsection{Optik}
Newton hielt seine Antrittsvorlesungen über seine Theorie der Farben. Als die Royal Society von seinem Spiegelteleskop erfuhr, konnte er es dort vorführen und stieß auf lebhaftes Interesse. In einem Brief an die Royal Society erwähnte er im Zusammenhang mit dem Bau des neuartigen Teleskops gegenüber dem damaligen Sekretär Henry Oldenburg eine neue Theorie des Lichtes. Das Ergebnis war die Veröffentlichung seiner Theorie über das Licht und die Farben,[11] die 1704 die Grundlage für das Hauptwerk Opticks or a treatise of the reflections, refractions, inflections and colours of light bildete („Optik oder eine Abhandlung über die Reflexion, Brechung, Krümmung und die Farben des Lichtes“).

Seit Johannes Keplers Schrift Paralipomena war die Optik ein zentraler Bestandteil der wissenschaftlichen Revolution des 17. Jahrhunderts. Ähnlich wie die Untersuchungen Galileo Galileis auf dem Gebiet der Mechanik hatte René Descartes’ Entdeckung des Gesetzes der Lichtbrechung die Ansicht untermauert, dass der Kosmos insgesamt nach mathematischen Grundsätzen angelegt sei. Abweichend von der antiken Vorstellung, farbige Erscheinungen beruhten auf einer Veränderung des Lichtes (das von Natur aus weiß sei), kam Newton durch Experimente mit Lichtspalt und Prisma zu dem Ergebnis, dass weißes Licht zusammengesetzt ist und durch das Glas in seine Farben zerlegt wird. (Vorläufer hatten behauptet, das Prisma füge die Farben hinzu.) Auf diese Weise konnte er mühelos die Entstehung des Regenbogens erklären.

Als Robert Hooke einige seiner Ideen kritisierte, war Newton so empört, dass er sich aus der öffentlichen Diskussion zurückzog. Die beiden blieben bis zu Hookes Tod erbitterte Kontrahenten.


Newtons Spiegelteleskopmodell von 1672 für die Royal Society (Nachbildung).
Aus seiner Arbeit schloss Newton, dass jedes mit Linsen aufgebaute Fernrohr unter der Dispersion des Lichtes leiden müsse, und schlug ein Spiegelteleskop vor, um die Probleme zu umgehen. 1672 baute er ein erstes Exemplar (siehe Abb.). Der von ihm vorgeschlagene (und später nach ihm benannte) Typ wurde für viele Generationen das Standardgerät für Astronomen. Allerdings war Newtons Prototyp den damals gebräuchlichen Linsenteleskopen nicht überlegen, da sein Hauptspiegel nicht parabolisiert war und daher unter sphärischer Aberration litt. Später wurden achromatische Linsenkombinationen aus Gläsern verschiedener Brechungseigenschaften für Fernrohre entwickelt.

Seine Feststellung, dass einzelne Lichtstrahlen unveränderliche Eigenschaften haben, führte ihn zu der Überzeugung, Licht bestehe aus (unveränderlichen und atomähnlichen) Lichtteilchen. Damit wich er grundlegend von Descartes ab, der Licht als Bewegung in Materie beschrieben hatte und weißes Licht als ursprünglich (und sich damit nicht so weit von Aristoteles entfernt hatte). Nach Newton entsteht der Eindruck der Farben durch Korpuskeln unterschiedlicher Größe.

In der Schrift Hypothesis of Light von 1675 führte Newton das Ätherkonzept ein:[12] Lichtpartikel bewegen sich durch ein materielles Medium – dies war reiner Materialismus. Unter dem Einfluss seines Kollegen Henry More ersetzte er den Lichtäther jedoch bald durch – aus dem hermetischen Gedankengut stammende – okkulte Kräfte, die die Lichtpartikel anziehen bzw. abstoßen.

Mit der Teilchentheorie des Lichtes waren allerdings Phänomene wie die – von Newton selbst beschriebene und genutzte – Interferenz oder die Doppelbrechung (auf Grund von Polarisation, von Erasmus Bartholin bereits im Jahr 1669 beschrieben) nicht erklärbar.

In der New Theory about Light and Colours vertrat Newton neben seiner Farb- auch seine Korpuskeltheorie. Dies führte zu einem wiederum erbittert ausgetragenen Disput mit Christiaan Huygens und dessen Wellentheorie des Lichtes, welchen er 1715 durch Desaguliers vor der Royal Society für sich entscheiden ließ. Nachdem Thomas Young im Jahre 1800, lange nach beider Tod, weitere Experimente zur Bestätigung der Wellentheorie durchgeführt hatte, wurde diese zu herrschenden Lehre. Heute sind beide Theoriekonzepte in der Quantenmechanik mathematisch vereint – wobei allerdings das moderne Photonenkonzept mit Newtons Korpuskeln kaum etwas gemeinsam hat.

%%%%%%%%%%%%%% Mechanik %%%%%%%%%%%%%%%%%

\subsection{Mechanik}
Auch die Grundsteine der klassischen Mechanik, die drei Grundgesetze der Bewegung und die Konzepte von absoluter Zeit, absolutem Raum und der Fernwirkung (und so auch indirekt das Konzept des Determinismus) wurden von ihm gelegt. Zusammen waren dies die wesentlichen Grundprinzipien der Physik seiner Zeit. Newton lehrte eine dualistische Naturphilosophie – beruhend auf der Wechselwirkung von aktiven immateriellen „Naturkräften“ mit der absolut passiven Materie –, welche zur Basis des naturwissenschaftlichen Weltbildes vieler Generationen wurde. Erst die Relativitätstheorie Albert Einsteins machte deutlich, dass Newtons Mechanik einen Spezialfall behandelt.

Vom Jahr 1678 an beschäftigte er sich, in Zusammenarbeit mit Hooke und Flamsteed, wieder intensiv mit Mechanik, insbesondere mit den von Kepler formulierten Gesetzen. Seine vorläufigen Ergebnisse veröffentlichte er 1684 unter dem Titel De Motu Corporum. In diesem Werk ist allerdings noch nicht die Rede von der universellen Wirkung der Schwerkraft; auch seine drei Gesetze der Bewegung werden hier noch nicht dargelegt. Drei Jahre später erschien, dieses Mal mit Unterstützung von Edmond Halley, die Zusammenfassung Philosophiae Naturalis Principia Mathematica (Mathematische Grundlagen der Naturphilosophie). Mit diesem Werk wollte er insbesondere die Naturphilosophie von Descartes ablösen (Principia philosophiae, 1644), obwohl er von diesem das Konzept der Trägheit übernehmen musste, das ein Zentralpunkt der newtonschen Mechanik wurde.

Newton war der Erste, der Bewegungsgesetze formulierte, die sowohl auf der Erde wie auch am Himmel gültig waren – ein entscheidender Bruch mit den Ansichten der traditionellen Lehre von Aristoteles und späterer Peripatetiker, wonach die Verhältnisse im Himmel grundlegend andere seien als auf der Erde. Darüber hinaus lieferte er die geometrische Argumentation für Keplers drei Gesetze, führte sie auf einheitliche Ursachen (Fernwirkung der Gravitation und Trägheit) zurück und erweiterte sie dahingehend, dass nicht nur Ellipsen, sondern sämtliche Kegelschnitte möglich seien (Georg Samuel Dörffel hatte allerdings bereits 1681 gezeigt, dass Kometen sich auf hyperbolischen Bahnen bewegen). Mit seinen drei Bewegungsgesetzen und der Einführung der allgemein wirkenden Schwerkraft (auch das Wort Gravitation geht auf ihn zurück) hatte Newton die Arbeiten von Kopernikus, Kepler und Galilei überzeugend bestätigt.

Seine Mechanik galt Generationen von Wissenschaftlern und Historikern als fundamentaler Beitrag im Sinne rationaler Begründung von Naturgesetzen (hypotheses non fingo bedeutet sinngemäß: „In der Experimentalphilosophie gibt es keine Unterstellungen“). Dabei wird gerne übersehen, dass Newtons Überlegungen auf einem Konzept beruhten, das durchaus nicht als objektiv wissenschaftlich gilt: der hermetischen Tradition, mit der er sich während der Quarantänezeit 1665–1666 eingehend beschäftigt hatte. Die traditionelle Naturphilosophie erklärte Naturerscheinungen mit der Bewegung materieller Teilchen (so etwa statische Elektrizität) durch ein ätherartiges Medium (so noch Newtons Hypothesis of Light von 1675). Eine Fernwirkung (durch „Kräfte“) erschien ihr ebenso unmöglich wie das Vakuum. So findet sich sowohl bei Descartes wie bei Leibniz (1693) die Vorstellung, dass Wirbel in einem „Fluidum“ (Lateinisch für Flüssigkeit) die Planeten auf ihren Bahnen hielten. Von 1679 an jedoch schrieb Newton gewisse Vorgänge (exotherme Reaktion oder Oberflächenspannung) der Wirkung anziehender bzw. abstoßender Kräfte zu – dies war eine direkte Umsetzung der okkulten „Sympathien“ bzw. „Antipathien“ der hermetischen Naturphilosophie. Wesentlich neu war jedoch, dass Newton diese Kräfte als Quantitäten behandelte, die sich sowohl experimentell als auch mathematisch-geometrisch fassen lassen.

1679 suchte Hooke den Kontakt mit Newton zu erneuern und erwähnte in einem Brief seine Theorie der Planetenbewegung. Darin war die Rede von einer Anziehungskraft, die mit der Entfernung abnimmt; Newtons Antwort ging von konstanter Schwerkraft aus. Dieser Briefwechsel (der sich mit einem Experiment auf der Erde befasste) war Ausgangspunkt des späteren Plagiatsvorwurfs von Hooke an Newton. Newton musste zugeben, dass Hooke ihn auf den richtigen Weg geführt habe: 1. eine Bahnellipse rührt von einer (mit dem Quadrat der Entfernung von einem Brennpunkt) abnehmenden Anziehungskraft her und 2. erklärt dieses Konzept außerirdische, also planetarische Bewegung. Jedoch beruhte Hookes Vorschlag abnehmender Schwerkraft auf Intuition, nicht – wie bei Newton – auf Beobachtung und logischer Ableitung. Außerdem hatte Newton selbst das Konzept quadratisch abnehmender Schwerkraft bereits 1665/66 entwickelt. Andererseits kam Newton auf den Gedanken der universellen (also auch außerirdischen) Wirkung der Schwerkraft erst deutlich nach 1680.

Es wird auch die Geschichte erzählt, dass Isaac Newton durch die Betrachtung eines Apfels am Apfelbaum, evtl. auch des Falls des Apfels vom Baum, im Garten von Woolsthorpe Manor auf die Idee kam, die Himmelsmechanik beruhe auf derselben Gravitation wie der Fall von Äpfeln auf die Erde. Dies geht auf die Memoires of Sir Isaac Newton’s Life von William Stukeley zurück; mit ähnlichen Worten schilderte Voltaire die legendäre Entdeckung. Ob es sich wirklich so zugetragen hat, bleibt fraglich. Fachleute halten es für möglich, dass Newton selbst in späteren Jahren die Geschichte erfunden hat, um darzulegen, wie er Einsichten aus Alltagsbeobachtungen gewonnen habe.

Die geometrisch orientierten Darlegungen Newtons in den Principia waren nur Fachleuten verständlich. Daran änderten auch zwei spätere Ausgaben (1713 mit wesentlichen Erweiterungen und 1726) nichts. Der Durchbruch auf dem Kontinent ist Émilie du Châtelet zu verdanken, die von 1745 an das Werk in Französische übersetzte, die geometrische Ausdrucksweise Newtons in die von Leibniz entwickelte Notation der Infinitesimalrechnung übertrug und seinen Text mit zahlreichen eigenen Kommentaren ergänzte.



%%%%%%%%%%%%%%%%%%%%%%%%%%%%%%%


\section{Formel}
$$ P(A) = \sum P(\{ (e_1,...,e_N) \})  =  {N}\choose{k} \cdot p^kq^{N-k}$$


%%%%%%%%%%%%%%%%%%%%%%%%%%%%%%%


\section{Tabelle}
\begin{center}
\begin{tabular}{ |c|c|c| } 
 \hline
 newton & newton & newton \\ 
 newton & newton & newton \\ 
 newton & newton & newton \\ 
 \hline
\end{tabular}
\end{center}


%%%%%%%%%%%%%%%%%%%%%%%%%%%%%%%

