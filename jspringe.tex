\documentclass{scrartcl}
\usepackage[utf8]{inputenc}
\usepackage{hyperref} 

\title{Aufgabe 12}
\author{Jan Springer}
\date{February 2021}

\begin{document}

\maketitle

\section{One, two, don't know what to do...}
das ist von Helge Schneider

\section{Sagt der Sohn zum Pharao, komm gehen wir aufs Damenklo!}
Das steht auf dem Herrenklo im Philosophikum, Südausgang.

\section{Tell my Y do Z Germans can't say Zee}
because of ze dental fricative stuff said the voiceless voice.

\begin{table}[]
    \centering
    \begin{tabular}{cccc|c}
    \hline
        1 & 2 & 3 & 4 & A \\
        5 & 6 & 7 & 8 & B \\ \hline
        6 & 8 & 10 & 12& C \\
    \end{tabular}
    \caption{Caption}
    \label{tab:my_label}
\end{table}

\section{ Some formulas }
\[ x^n + y^n = z^n \] 
\newline
\[ A \cup B = B \cup A, A \cap B = B \cap A \] % \[\] oder $ beeinflußen die Einrückung..
\\
$ A \cup B = B \cup A, A \cap B = B \cap A $
\\*

 $\sum\limits_{i=0}^{n}$ % Dies ist ein Kommentar...

\section{Corona Artikel von Telepolis.de}





 Die Corona-Infektionszahlen gehen zurück; die Gereiztheit steigt. Das ganze diskutierende System steckt in einer erhöhten Nervosität. Man muss nur einen Blick in das hiesige Forum werfen, um einen Eindruck zu bekommen. Die einstmaligen Klagen über die Konsensrepublik Deutschland sind Schnee von vorgestern.

"Gesellschaften werden durch geteilten Stress gebildet", stellte Peter Sloterdijk vor zwei Jahren fest. Schon damals, weit vor Corona, ging es um hypernervöse Gemeinschaften, zunehmende Aggressivität und eine Welt aus den Fugen.

Am morgigen Mittwoch gibt es ein neues Bund-Länder-Treffen zur augenblicklichen Infektions-Lage. Jüngste Ankündigungen dämpfen die Erwartungen, dass es zu Lockerungen der Maßnahmen kommt. In Bayern sprach sich etwa der Mann, der als "erster CSU-Kanzler" (Albrecht von Lucke) Geschichte machen könnte, gegen Lockerungen aus. Söder sieht "keine Alternative zur Lockdown-Verlängerung".
\\
\begin{addmargin}[25pt]{0pt}
    Ich glaube, grundsätzlich wird der Lockdown erstmal verlängert werden müssen. Es hat ja keinen Sinn, jetzt abzubrechen einfach.
\\*    
 \textit{Markus Söder}   
\\    
\end{addmargin}
\newline
\url{https://www.heise.de/tipps-tricks/} zeigt wie man ein Link setzt.



\end{document}
