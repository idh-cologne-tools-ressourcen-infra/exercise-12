\documentclass{scrartcl}
\usepackage[utf8]{inputenc}
\usepackage{amsmath}
\title{Exercise 12}
\author{twerneck }
\date{February 2021}

\begin{document}

\maketitle

\tableofcontents

\section{Das Sofa Problem}
Quelle: Wikipedia

Das Sofaproblem ist ein bislang ungelöstes geometrisches Problem, das 1966 vom österreichisch-kanadischen Mathematiker Leo Moser beschrieben wurde. Es ist eine zweidimensionale Idealisierung des praktischen Problems, Möbelstücke um Hindernisse zu bewegen.

Das Sofaproblem ist die Frage, welches die zweidimensionale, starre Form mit der größten Fläche A ist, die innerhalb eines L-förmigen Korridors der Breite 1 um die rechtwinklige Ecke manövriert werden kann. Die Fläche A wird als Sofakonstante bezeichnet.


\subsection{"Obere und untere Schranken"}
Ein Halbkreis mit Radius 1 kann um die Ecke herum transportiert werden, indem man den Halbkreis zunächst gerade bis zur Begrenzung durchschiebt. Nun fällt die Ecke mit dem Mittelpunkt der Grundseite des Halbkreises zusammen, und man kann den Halbkreis um diesen Punkt herum um 90° drehen. Anschließend kann man den Halbkreis weiterschieben. Eine erste untere Grenze für den Flächeninhalt ist demnach:

$ A = Pi / 2  1,57 $



... eigentlich geht es hier noch weiter, aber ich frage mich grade, ob die Formel so richtig angezeigt wird ...


\section{Gedanken zum Problem}
Wer kennt es nicht, das gute alte Sofa-Problem. Bei jedem Umzug eine neue Herausforderung. 


\begin{tabular}{c|c|c}
    \hline
        &  Sofa & Bea \\
    \hline
    Umzug 1 & 1 & 0\\
    
    \hline
     Umzug 2 & 0 & 1
    
\end{tabular}


\section{Genug Text um zwei Seiten zu schaffen, für ein Inhaltsverzeichnis}
Der Wikipedia Artikel des Tages: 09.02.2021
Marmor

\subsection{Entstehung}
Marmor entsteht durch metamorphe Umwandlung von Kalksteinen, Dolomiten und anderen carbonatreichen Gesteinen unter Einfluss von hohem Druck und hoher Temperatur infolge hoher Sedimentsauflast und/oder tektonischer Versenkung (Regionalmetamorphose) oder durch Aufheizung im Kontakt mit Gesteinsschmelze (Kontaktmetamorphose). Sind Dolomite umgewandelt worden, spricht man von einem Dolomitmarmor.

Bei der Kontaktmetamorphose intrudieren granitische oder andere Magmen in die obere Erdkruste. Falls sie die Erdoberfläche nicht erreichen, verbleiben sie in der Erdkruste, kühlen in Magmenkammern über Jahrtausende ab und erstarren zu Granit oder magmatischen Gesteinen ähnlicher Zusammensetzung. Während dieser Phase der Abkühlung können sich karbonatreiche Gesteine in der Umgebung des Granitplutons zu Marmor umwandeln. Bei einer Kontaktmetamorphose herrschen Drücke bis 10 Kilobar und Temperaturen über 400 °C.[4]

Bei der Regionalmetamorphose werden große Mengen an Gestein unter Druck und Hitze ohne Magmenkontakt umgewandelt. Diese Prozesse laufen sehr langsam ab. Dabei können zum Beispiel Marmore mit Richtungsgefüge (spaltraue Platten gewinnbar) entstehen. Die bevorzugte Spaltrichtung liegt meist orthogonal zur Richtung der früheren Hauptspannung. Da sich Marmore ab einem bestimmten Druck- und Temperaturniveau duktil verformen, können sie Falten und Fließgefüge zeigen, die bei inhomogener Verteilung der Nebengemengebestandteile als Marmorierung sichtbar sind (z. B. im Saillon-Marmor von Saillon VS, Schweiz). Duktil bedeutet in der Geologie, dass sich Gesteine insbesondere der unteren kontinentalen Erdkruste unter tektonischem Stress (Hitze und Druck) nicht spröde, sondern plastisch deformieren.



\end{document}
