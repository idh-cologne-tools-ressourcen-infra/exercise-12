\documentclass{article}  
\usepackage[utf8]{inputenc}  
  
\title {Film und Serien}  
\author {Kerstin van Kan}  
\date{February 2021}  
  
\begin{document}  
  
\maketitle  
\tableofcontents  
  
  
\newpage  
\section {Harry Potter}  
Harry Potter ist eine Kinder- und Jugendromanreihe der englischen Schriftstellerin Joanne K. Rowling. Erzählt wird die Geschichte der titelgebenden Figur Harry Potter, der an seinem elften Geburtstag von seiner magischen Herkunft erfährt und fortan Schüler des britischen Zaubererinternats Hogwarts ist. Jeder der sieben Bände beschreibt ein Schul- und Lebensjahr von Harry Potter und seinen Freunden. Die Buchreihe inklusive drei Ableger wurde weltweit über 500 Millionen Mal verkauft und in 80 Sprachen übersetzt.[1] Allein in deutscher Sprache wurden über 33 Millionen Harry-Potter-Bücher verkauft.[2]  
  
Der erste Teil Harry Potter und der Stein der Weisen wurde am 26. Juni 1997 in einer Erstauflage von 500 Stück bei Bloomsbury Publishing veröffentlicht.[3] Der zweite Teil Harry Potter und die Kammer des Schreckens folgte 1998 und der dritte Teil Harry Potter und der Gefangene von Askaban im Jahr 1999. Ab diesem Zeitpunkt belegten die Bücher in zahlreichen Ländern die obersten Plätze der Bestsellerlisten, so beispielsweise in The New York Times[4] oder Der Spiegel. Mit der Startauflage des vierten Teils Harry Potter und der Feuerkelch im Jahr 2000 stellte die Reihe einen Rekord auf. Auch die Bände fünf bis sieben brachen Vorbestellungsrekorde u. a. in Großbritannien und den USA.  
  
In den Jahren 2001 bis 2011 wurden die Bücher in acht Teilen verfilmt; die Filmreihe gilt als eine der kommerziell erfolgreichsten der Filmgeschichte. Die Bücher und die Filme lösten in den späten 1990ern und den frühen 2000er Jahren eine weltweite Begeisterung für Fantasy-Bücher und -Filme aus, einhergehend mit (Neu-)Verfilmungen von Der Herr der Ringe oder Die Chroniken von Narnia sowie zahlreichen weiteren Fantasy-Buchreihen wie Eragon, Bartimäus oder Percy Jackson.  
  
Die Harry-Potter-Heptalogie kann mehreren literarischen Genres zugeordnet werden. Sie hat Eigenschaften eines Kriminalromans, Entwicklungsromans und eines Bildungsromans, kann aber auch als modernes Märchen bzw. Kunstmärchen gesehen werden. Einhergehend mit der anhaltenden Begeisterung für die Harry-Potter-Bücher und -Filme wurden zahlreiche Merchandising-Produkte mit großem Erfolg vermarktet und ein eigener Themenpark eröffnet.[5]  
  
\newpage  
\section {Herr der Ringe}  
Der Herr der Ringe (englischer Originaltitel: The Lord of the Rings) ist ein Roman von John Ronald Reuel Tolkien. Er gehört zu den kommerziell erfolgreichsten Romanen des 20. Jahrhunderts, ist ein Klassiker der Fantasy-Literatur und gilt als grundlegendes Werk der High Fantasy. Im englischen Original in drei Bänden zu jeweils zwei Büchern plus Appendizes in den Jahren 1954/1955 veröffentlicht, erschien die erste deutsche Übersetzung von Margaret Carroux 1969/1970. Weltweit wurde der Roman etwa 150 Millionen Mal verkauft.[1]  
  
Der Roman steht vor dem Hintergrund einer von Tolkien sein Leben lang entwickelten Fantasiewelt (Tolkiens Welt). Er erzählt die Geschichte eines Rings, mit dessen Vernichtung die böse Macht in Gestalt des dunklen Herrschers Sauron untergeht. Die Hauptfiguren sind vier Hobbits, die unfreiwillig in ein heroisches Abenteuer hineingezogen werden. Neben diesen spielen als Vertreter des Guten Elben, Menschen des Westens und des Nordens, Zwerge und Zauberer wichtige Rollen. Ihre Gegenspieler sind die Geschöpfe und Untertanen Saurons, die Orks, Trolle und Menschen des Ostens und des Südens.  
  
Der Roman diente als Vorlage für zahlreiche Adaptionen, darunter eine mit 17 Oscars prämierte Filmtrilogie (2001–2003).  
  
\newpage  
\section {Game of Thrones}  
Game of Thrones (engl. für „Spiel der Throne“, oft abgekürzt mit GoT) ist eine US-amerikanische Fantasy-Fernsehserie von David Benioff und D. B. Weiss für den US-Kabelsender HBO. Die von Kritikern gelobte und kommerziell erfolgreiche Serie basiert auf der Romanreihe A Song of Ice and Fire („Das Lied von Eis und Feuer“) des US-amerikanischen Schriftstellers George R. R. Martin, der anfangs ebenfalls an der Serie mitwirkte. Die späteren Staffeln weichen allerdings zunehmend von der Buchvorlage ab.  
  
Die Handlung ist in einer fiktiven Welt angesiedelt und spielt auf den Kontinenten Westeros und Essos. Die sieben Königreiche von Westeros ähneln dem europäischen Mittelalter und sind durch eine riesige Mauer aus Eis von einem Gebiet des ewigen Winters im Norden abgeschirmt. Frühling, Sommer, Herbst und Winter können mehrere Jahre oder gar Jahrzehnte andauern, ihre Länge ist variabel und nicht vorhersehbar. Die Geschichte beginnt am Ende eines langen Sommers und wird zunächst in drei Handlungssträngen weitestgehend parallel erzählt. Zwischen den mächtigen Adelshäusern des Reiches bauen sich Spannungen auf, die schließlich zum offenen Thronkampf führen. Der Winter bahnt sich an, und es zeichnet sich im hohen Norden von Westeros, wo die Nachtwache an der Mauer die Königreiche schützen soll, eine Gefahr durch eine fremde Macht ab. In Essos ist Daenerys Targaryen als Erbin der vor Jahren abgesetzten Königsfamilie von Westeros bestrebt, wieder an die Macht zu gelangen.  
  
Die komplexe Handlung umfasst zahlreiche Figuren und thematisiert unter anderem Politik und Machtkämpfe, Gesellschaftsverhältnisse, Kriege und Religionen. Zahlreiche Charaktere weisen differenzierte moralische Facetten auf; ebenso kommen im Verlauf der Serie einzelne Protagonisten ums Leben.  
  
Die Erstausstrahlung erfolgte am 17. April 2011 bei HBO, gefolgt von der deutschsprachigen Erstausstrahlung am 2. November 2011.  
  
Insgesamt besteht die Serie aus acht Staffeln, wobei die ersten sechs Staffeln jeweils aus zehn, die siebte Staffel aus sieben und die achte Staffel aus sechs Episoden besteht. Die letzte Folge der achten Staffel und somit der gesamten Serie wurde am 19. Mai 2019 in den Vereinigten Staaten und am 20. Mai 2019 in Deutschland erstausgestrahlt.  
  
  
\newpage  
\begin{equation}  
p(x) = 8x^12 + 1x^9y - 2x^2 + 187x^3y\\   
- 26x^3y^4 - 12xy^5   
\end{equation}   
  
\begin{center}  
\begin{tabular}{ l | c | r }  
  \hline			  
   Hamburg & Berlin & München \\  
  Australien & Europa & Asien \\  
  Deutschland & Russland & Türkei \\  
  \hline    
\end{tabular}  
\end{center}  
  
  
\end{document}  
