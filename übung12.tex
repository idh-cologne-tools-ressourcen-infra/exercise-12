\documentclass{scrartcl}
\usepackage[utf8]{inputenc}
\usepackage[german]{babel}
\usepackage{blindtext}
\usepackage{microtype}
\usepackage{enumitem}
\usepackage{fancyhdr}
\usepackage{amsmath}
\usepackage{index}
\usepackage{emoji}
\renewcommand{\familydefault}{\sfdefault}
\title{Übung 12 - Infrastruktur}
\author{David Wegmann }
\date{February 2021}

\begin{document}

\maketitle
\tableofcontents
\newpage

\pagenumbering{roman}
\setcounter{page}{1}
\fancyhf{}


\section{Test}
\blindmathtrue
\blindtext[3]
\subsection{Test2}
\blindtext[2]
\begin{table}
\begin{center}
\begin{tabular}{ c c c }
\emoji{rocket} & \emoji{rocket} & \emoji{rocket} \\ 
GME & AMC & BB \\  
\emoji{rocket} & \emoji{rocket} & \emoji{rocket}

\end{tabular}
\caption{To the Moooooon}
\label{tab:Stocks only go up}
\end{center}
\end{table}
\section{Stonks}
\blindtext[4]
\section{Test 3}
\begin{table}[]
    \centering
    \begin{tabular}{c|c}
    $\displaystyle\pi(x) = \sum_{n = 1}^\infty \frac{\mu(n)}{n} J(\sqrt[n]{x})$     & $a²+b²=c²$  
    \end{tabular}
    \caption{Ein paar Random Gleichungen (Formel für die Primzahlzählfunktion + Satz des Pythagoras) }
    \label{tab:my_label}
\end{table}
\blindtext[5]
\end{document}
